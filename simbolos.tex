\makeatletter

\usepackage{amsthm, amsmath, amssymb, bm, mathtools}
\usepackage{enumitem, etoolbox, xpatch}
% \usepackage[mathcal]{euscript}
% \usepackage[scr]{rsfso}
\usepackage{mathptmx}
\usepackage{relsize, centernot, tikz, xcolor}


%%%% QED symbols %%%%
\def\qed@open{\ensuremath{\square}}
\def\qed@open@small{\ensuremath{\mathsmaller\qed@open}}

\def\qed@fill{\ensuremath{\blacksquare}}
\def\qed@fill@small{\ensuremath{\mathsmaller\qed@fill}}

\definecolor{qed@gray}{gray}{0.8}
\def\qed@gray{\ensuremath{\color{qed@gray}\blacksquare}}
\def\qed@gray@small{\ensuremath{\color{qed@gray}\mathsmaller\blacksquare}}

\def\qed@both@cmd#1#2{\begin{tikzpicture}[baseline=#2]
    \draw (0,0) [fill=qed@gray] rectangle (#1,#1);
\end{tikzpicture}}
\def\qed@both{\qed@both@cmd{0.6em}{0.2ex}}
\def\qed@both@small{\qed@both@cmd{1ex}{0ex}}

\def\showAllQED{
    \qed@open ~ \qed@open@small \\
    \qed@fill ~ \qed@fill@small \\
    \qed@gray ~ \qed@gray@small \\
    \qed@both ~ \qed@both@small
}

%% escoha do QED %%
\renewcommand{\qedsymbol}{\qed@fill@small}

% marcadores de prova
\newcommand{\direto}[1][~]{\ensuremath{(\rightarrow)}#1}
\newcommand{\inverso}[1][~]{\ensuremath{(\leftarrow)}#1}

% fontes
% conjunto potencia
\DeclareSymbolFont{boondox}{U}{BOONDOX-cal}{m}{n}
\DeclareMathSymbol{\pow}{\mathalpha}{boondox}{"50}

% somatorio
\DeclareSymbolFont{matext}{OMX}{cmex}{m}{n}
\DeclareMathSymbol{\sum@d}{\mathop}{matext}{"58}
\DeclareMathSymbol{\sum@t}{\mathop}{matext}{"50}
\undef\sum
\DeclareMathOperator*{\sum}{\mathchoice{\sum@d}{\sum@t}{\sum@t}{\sum@t}}
\DeclareMathOperator*{\bigsum}{\mathlarger{\mathlarger{\sum@d}}}

% phi computer modern
\DeclareMathAlphabet{\gk@mf}{OT1}{cmr}{m}{n}
\let\old@Phi\Phi
\def\Phi{\gk@mf{\old@Phi}}

% união elem por elem
\DeclareMathOperator{\wcup}{\mathaccent\cdot\cup}

% familia de conjuntos
\undef\fam
\DeclareMathAlphabet{\fam}{OMS}{cmsy}{m}{n}

% alguns símbolos
\undef\natural
\DeclareMathOperator{\real}{\mathbb{R}}
\DeclareMathOperator{\natural}{\mathbb{N}}
\DeclareMathOperator{\integer}{\mathbb{Z}}
\DeclareMathOperator{\complex}{\mathbb{C}}
\DeclareMathOperator{\rational}{\mathbb{Q}}
\def\symdif{\mathrel{\triangle}}
\def\midd{\;\middle|\;}
% \def\pow{\mathcal{P}}

% operações com mais espaçamento
\def\cupp{\mathbin{\,\cup\,}}
\def\capp{\mathbin{\,\cap\,}}

% alguns operadores
\DeclareMathOperator{\Dom}{Dom}
\DeclareMathOperator{\Img}{Im}

% marcadores de operadores
\def\inv{^{-1}}
\def\rel#1{\use@invr{\!\mathrel{#1}\!}}
\def\nrel#1{\use@invr{\!\centernot{#1}\!}}
\def\dmod#1{\ (\mathrm{mod}\ #1)}
\def\cgc#1{{\textnormal{[}#1\textnormal{]}}}
\def\cgp#1{{\textnormal{(}#1\textnormal{)}}}

% marcador com inverso reduzido
\def\use@invr#1{%
    \begingroup%
        \edef\inv{\inv\!}%
        #1%
    \endgroup%
}

% delimiters
\def\abs#1{{\lvert\,#1\,\rvert}}
% \DeclarePairedDelimiter{\abs}{\lvert}{\:\rvert}

\makeatother
