\documentclass[a4paper, 14pt]{extarticle}

\usepackage[brazilian]{babel}
% \usepackage[utf8]{inputenc}
\usepackage[T1]{fontenc}
\usepackage[margin=1.5cm,top=1.8cm,noheadfoot=true]{geometry}

\usepackage{float, pgf, caption, subcaption}

% \input{secoes}
% \input{teorema}
\makeatletter

\usepackage{amsthm, amsmath, amssymb, bm, mathtools}
\usepackage{enumitem, etoolbox, xpatch}
% \usepackage[mathcal]{euscript}
% \usepackage[scr]{rsfso}
\usepackage{mathptmx}
\usepackage{relsize, centernot, tikz, xcolor}


%%%% QED symbols %%%%
\def\qed@open{\ensuremath{\square}}
\def\qed@open@small{\ensuremath{\mathsmaller\qed@open}}

\def\qed@fill{\ensuremath{\blacksquare}}
\def\qed@fill@small{\ensuremath{\mathsmaller\qed@fill}}

\definecolor{qed@gray}{gray}{0.8}
\def\qed@gray{\ensuremath{\color{qed@gray}\blacksquare}}
\def\qed@gray@small{\ensuremath{\color{qed@gray}\mathsmaller\blacksquare}}

\def\qed@both@cmd#1#2{\begin{tikzpicture}[baseline=#2]
    \draw (0,0) [fill=qed@gray] rectangle (#1,#1);
\end{tikzpicture}}
\def\qed@both{\qed@both@cmd{0.6em}{0.2ex}}
\def\qed@both@small{\qed@both@cmd{1ex}{0ex}}

\def\showAllQED{
    \qed@open ~ \qed@open@small \\
    \qed@fill ~ \qed@fill@small \\
    \qed@gray ~ \qed@gray@small \\
    \qed@both ~ \qed@both@small
}

%% escoha do QED %%
\renewcommand{\qedsymbol}{\qed@fill@small}

% marcadores de prova
\newcommand{\direto}[1][~]{\ensuremath{(\rightarrow)}#1}
\newcommand{\inverso}[1][~]{\ensuremath{(\leftarrow)}#1}

% fontes
% conjunto potencia
\DeclareSymbolFont{boondox}{U}{BOONDOX-cal}{m}{n}
\DeclareMathSymbol{\pow}{\mathalpha}{boondox}{"50}

% somatorio
\DeclareSymbolFont{matext}{OMX}{cmex}{m}{n}
\DeclareMathSymbol{\sum@d}{\mathop}{matext}{"58}
\DeclareMathSymbol{\sum@t}{\mathop}{matext}{"50}
\undef\sum
\DeclareMathOperator*{\sum}{\mathchoice{\sum@d}{\sum@t}{\sum@t}{\sum@t}}
\DeclareMathOperator*{\bigsum}{\mathlarger{\mathlarger{\sum@d}}}

% phi computer modern
\DeclareMathAlphabet{\gk@mf}{OT1}{cmr}{m}{n}
\let\old@Phi\Phi
\def\Phi{\gk@mf{\old@Phi}}

% união elem por elem
\DeclareMathOperator{\wcup}{\mathaccent\cdot\cup}

% familia de conjuntos
\undef\fam
\DeclareMathAlphabet{\fam}{OMS}{cmsy}{m}{n}

% alguns símbolos
\undef\natural
\DeclareMathOperator{\real}{\mathbb{R}}
\DeclareMathOperator{\natural}{\mathbb{N}}
\DeclareMathOperator{\integer}{\mathbb{Z}}
\DeclareMathOperator{\complex}{\mathbb{C}}
\DeclareMathOperator{\rational}{\mathbb{Q}}
\def\symdif{\mathrel{\triangle}}
\def\midd{\;\middle|\;}
% \def\pow{\mathcal{P}}

% operações com mais espaçamento
\def\cupp{\mathbin{\,\cup\,}}
\def\capp{\mathbin{\,\cap\,}}

% alguns operadores
\DeclareMathOperator{\Dom}{Dom}
\DeclareMathOperator{\Img}{Im}

% marcadores de operadores
\def\inv{^{-1}}
\def\rel#1{\use@invr{\!\mathrel{#1}\!}}
\def\nrel#1{\use@invr{\!\centernot{#1}\!}}
\def\dmod#1{\ (\mathrm{mod}\ #1)}
\def\cgc#1{{\textnormal{[}#1\textnormal{]}}}
\def\cgp#1{{\textnormal{(}#1\textnormal{)}}}

% marcador com inverso reduzido
\def\use@invr#1{%
    \begingroup%
        \edef\inv{\inv\!}%
        #1%
    \endgroup%
}

% delimiters
\def\abs#1{{\lvert\,#1\,\rvert}}
% \DeclarePairedDelimiter{\abs}{\lvert}{\:\rvert}

\makeatother


% math display skip
\newcommand{\reducemathskip}[1][0.5em]{%
    \setlength{\abovedisplayskip}{1pt}%
    \setlength{\belowdisplayskip}{#1}%
    \setlength{\abovedisplayshortskip}{#1}%
    \setlength{\belowdisplayshortskip}{#1}%
}

% url linking problems
\def\url#1{\href{#1}{\texttt{#1}}}
% vermelho
\def\red#1{\textcolor{red}{#1}}

\def\lmref#1{\thmref[lema ]{#1}}

\usepackage{xparse, caption, booktabs}
\usepackage[hidelinks]{hyperref}
\usepackage[nameinlink, brazilian]{cleveref}
\crefformat{equation}{#2eq.~#1#3}
\crefformat{definition}{#2def.~#1#3}
\crefformat{proof}{#2dem.~#1#3}
\usepackage[section, newfloat]{minted}
\definecolor{sepia}{RGB}{252,246,226}
\setminted{
    bgcolor = sepia,
    % style   = pastie,
    frame   = leftline,
    autogobble,
    samepage,
    python3,
}
\setmintedinline{
    bgcolor={}
}

\theoremstyle{plain}
\newtheorem*{hypothesis}{Hipótese}
\newtheorem*{theorem}{Teorema}
\newtheorem*{hypothesisf}{Hipótese Fortalecida}

\newtheoremstyle{definicao}% name of the style to be used
  {}% measure of space to leave above the theorem. E.g.: 3pt
  {}% measure of space to leave below the theorem. E.g.: 3pt
  {}% name of font to use in the body of the theorem
  {}% measure of space to indent
  {\bf}% name of head font
  {:}% punctuation between head and body
  {.8em}% space after theorem head; " " = normal interword space
  {\thmnote{\textbf{#3}}}% Manually specify head
\theoremstyle{definicao}
\newtheorem*{definition}{Definição}

\NewDocumentCommand{\seq}{ s m O{n} O{\in\natural} }
    {\IfBooleanTF{#1}
        {\ensuremath{\left({#2}_{#3}\right)}}
        {\ensuremath{\left({#2}_{#3}\right)_{{#3}{#4}}}}}


\usepackage{titling, titlesec, enumitem}
% \usepackage{algorithmic}
\usepackage{clrscode3e, xspace}
\title{\vspace{-2.5cm}\Large Lista de Exercícios Avaliativa 5 \\ \normalsize MC458 - 2s2020 - Tiago de Paula Alves - 187679}
\preauthor{}\author{}\postauthor{}
\predate{}\date{}\postdate{}
\posttitle{\par\end{center}\vskip-1em}

\titleformat{\section}{\large\bfseries}{\thesection}{.8em}{}
\titlespacing*{\section}{0pt}{.5em plus .2em minus .2em}{.5em plus .2em}

\newlist{casos}{enumerate}{2}
\setlist[casos]{wide,labelwidth={\parindent},listparindent={\parindent},parsep={\parskip},topsep={0pt},label=\textbf{Caso \arabic*}:}
\setlist[casos,2]{label=\textbf{Caso \arabic{casosi}\alph*}:}

\newlist{ncasos}{description}{2}
\setlist[ncasos]{wide,listparindent={\parindent},parsep={\parskip},topsep={0pt}}

\titleformat{\section}[runin]
    {\titlerule{}\vspace{1ex}\normalfont\Large\bfseries}{}{1em}{}[.]
\titleformat{\subsection}[runin]
    {\normalfont\large\bfseries}{}{1em}{}[)]

% linha final da página ou seção
\newcommand{\docline}[1][\\]{%
    #1\noindent\rule{\textwidth}{0.4pt}%
    \pagebreak%
}
\newcommand{\itemdsep}{
    \noindent\hfil\rule{0.5\textwidth}{.2pt}\hfil
    \vskip1em
}


\usepackage{tikz}
\usetikzlibrary{calc,trees,positioning,arrows,fit,shapes,calc}

\DeclareMathSymbol{\mlq}{\mathord}{operators}{``}
\DeclareMathSymbol{\mrq}{\mathord}{operators}{`'}
\def\gets{~\leftarrow~}

\usepackage{fancyhdr}
\pagestyle{empty}

% \usepackage{showframe}
\begin{document}

    \maketitle
    \thispagestyle{empty}

    % \noindent\rule{\textwidth}{0.4pt}
    % \begin{center}\Large\vskip-0.5em
    %     CORRIGIR A QUESTÃO \textbf{???}.
    % \end{center}

    \section{1}
    \begingroup
        \subsection{a} Considere a operação ``$\oplus$'' como a concatenação de cadeias e a notação de prefixo adotada em aula.

\begin{theorem}[subestrutura ótima]
    Sejam $n$ e $m$ inteiros positivos e considere as cadeias de caracteres de $\Sigma$ sendo $X = \langle x_1, \ldots, x_m \rangle$, $Y = \langle y_1, \ldots, y_n \rangle$ e $Z = \langle z_1, \ldots, z_{m + n} \rangle$. Então, $Z$ é é uma intercalação de $X$ e $Y$, se e somente se uma das condições a seguir for atendida:

    1. $x_m = z_{n + m}$ ~e~ $Z_{m + n - 1}$ é intercalação $X_{m - 1}$ e $Y$.

    2. $y_n = z_{n + m}$ ~e~ $Z_{m + n - 1}$ é intercalação $X$ e $Y_{n - 1}$.
\end{theorem}

\begin{proof}[\textbf{Demonstração}]
    Considere $Z' = Z_{m + n - 1}$, $X' = X_{m - 1}$ e $Y' = Y_{n - 1}$.

    ~

    ($\rightarrow$) Suponha que $Z$ é uma intercalação de $X$ e $Y$ e . Suponha ainda que a condição 1 seja falsa. Então, temos os seguintes casos.

    \begin{casos}
        \item $x_m \ne z_{m + n}$. Assim, se $y_n \ne z_{m + n}$, então $Z$ não poderia ser uma intercalação de $X$ e $Y$, logo, $y_n = z_{m + n}$. Suponha agora que $Z'$ não é uma intercalação de $X$ e $Y'$, ou seja, não é possível decompor $Z'$ em subcadeias disjuntas iguais a $X$ e $Y'$.

        Para que exista uma subcadeia $S$ de $Z = Z' \oplus \langle z_{m + n} \rangle$ tal que $S = Y' \oplus \langle y_n \rangle = Y$, temos que $z_{m + n}$ deverá ser parte de $S$. Se $X$ não é subcadeia de $Z'$, como $z_{m + n}$ seria parte da subcadeia igual $Y$, então $X$ também não seria subcadeia de $Z$. Por outro lado, se $Y'$ não é subcadeia de $Z'$, então $Y = Y' \oplus \langle y_n \rangle$ também não poderá ser subcadeia de $Z$.

        Portanto, temos que $Z'$ deve ser uma intercalação de $X$ e $Y'$, ou seja, a condição 2 deve ser atendida.

        \item $x_m = z_{m + n}$ e $Z'$ não é intercalação de $X'$ e $Y$. Suponha que $z_{m + n}$ parte da subcadeia $S$ de $Z$ que é igual a $X$. No entanto, como $Z'$ não é intercalação de $X'$ e $Y$, temos que $Z = Z' \oplus \langle z_{m + n} \rangle$ não pode ser intercalação de $X = X' \oplus \langle x_m \rangle = S$ e $Y$. Portanto, $z_{m + n}$ deve ser parte de $Y$.

        Como $z_{m + n}$ é o último elemento de $Z$, então $y_n = z_{m + n}$. Além disso, assim como no caso anterior, temos que $Z'$ deve ser uma intercalação de $X$ e $Y'$.
    \end{casos}

    Por fim, temos que se a condição 1 é falsa, a condição 2 deve ser verdade. Logo, para que $Z$ seja a intercalação proposta, uma das condições deve ser atendida.

    ~

    ($\leftarrow$) Considere agora que a condição 1 é verdade. Como $Z = Z' \oplus \langle z_{m + n} \rangle$, $X' = X \oplus \langle x_m \rangle$ e $z_{m + n} = x_m$, então as operações de concatenação mantêm a propriedade intercalação. Portanto, $Z$ é uma intercalação de $X$ e $Y$.

    A condição 2 é similar.
\end{proof}

\itemdsep
\newpage

A partir da demonstração acima e considerando os casos onde $X$ ou $Y$ é vazio, podemos ver que a recorrência para verificação de intercalação será dada por:
\begin{align*}
    \varnothing &\text{ é intercalação de } \varnothing \text{ e } \varnothing \\
    Z \text{ é intercalação de } \varnothing \text{ e } Y &\bm{\leftrightarrow} z_n = y_n \text{ \textbf{e} } Z_{n - 1} \text{ é intercalação de } \varnothing \text{ e } Y_{n - 1} \\
    Z \text{ é intercalação de } X \text{ e } \varnothing &\bm{\leftrightarrow} z_m = x_m \text{ \textbf{e} } Z_{m - 1} \text{ é intercalação de } X_{m - 1} \text{ e } \varnothing \\
    Z \text{ é intercalação de } X \text{ e } Y &\bm{\leftrightarrow} \begin{cases}
        z_{m + n} = x_m \text{ \textbf{e} } Z_{m + n - 1} \text{ é intercalação de } X_{m - 1} \text{ e } Y &\text{, ou} \\
        z_{m + n} = y_n \text{ \textbf{e} } Z_{m + n - 1} \text{ é intercalação de } X \text{ e } Y_{n - 1}
    \end{cases}
\end{align*}

\itemdsep

\subsection{b} Considere $\const{VERDADEIRO}$ como 1, $\const{FALSO}$ como 0 e $\const{NULO}$ como uma constante diferente de 1 e de 0.

\def\And{\mbox{ \kw{e} }}
\def\Or{\mbox{ \kw{ou} }}

\begin{codebox}
    \Procname{$\proc{Intercalacao}(X, m, Y, n, Z)$}
    \li Seja $I[0 \twodots m][0 \twodots n]$ uma matriz
    \li
    \li \kw{para} $i = 0$ \kw{até} $m$
        \Do
    \li     \kw{para} $j = 0$ \kw{até} $n$
            \Do
    \li         $I[i][j] \gets \const{NULO}$
            \End
        \End
    \li
    \li \kw{retorna} $\proc{Intercalacao-PD}(X, m, Y, n, Z, I)$
\end{codebox}

\begin{codebox}
    \Procname{$\proc{Intercalacao-PD}(X, m, Y, n, Z, I)$}
    \li \kw{se} $I[m][n] \ne \const{NULO}$
        \Then
    \li     \kw{retorna} $I[m][n]$
        \End
    \li
    \li \kw{se} $m \isequal 0 \And n \isequal 0$
        \Then
    \li     $I[0][0] \gets \const{VERDADEIRO}$
        \End
    \li \kw{senão}
        \Then
    \li     $I[m][n] \gets \const{FALSO}$
    \li     \kw{se} $m \ne 0 \And X[m] \isequal Z[m + n]$
            \Then
    \li         $I[m][n] \gets I[m][n] \Or \proc{Intercalacao-PD}(X, m - 1, Y, n, Z, I)$
            \End
    \li     \kw{se} $n \ne 0 \And Y[n] \isequal Z[m + n]$
            \Then
    \li         $I[m][n] \gets I[m][n] \Or \proc{Intercalacao-PD}(X, m, Y, n-1, Z, I)$
            \End
        \End
    \li
    \li \kw{retorna} $I[m][n]$
\end{codebox}

\itemdsep
\newpage

\subsection{c}

Podemos ver que $\proc{Intercalacao-PD}$ faz no máximo duas chamadas recursivas, independentemente dos valores de $m$ ou $n$. Além disso, devido a memorização com a matriz $I$, a segunda chamada da função com mesmos valores de $m$ e $n$ é executada em tempo constante. Também é possível notar que cada chamada com valores $m'$ e $n'$, temos que $0 \leq m' \leq m$ e $0 \leq n' \leq n$.

Logo, no pior dos casos, as chamadas recursivas devem prencher todas as posições de $I$ com mais um número constante de operações. Portanto, a complexidade da função recursiva $\proc{Intercalacao-PD}$ é $O\left((m + 1) (n + 1)\right) = O(m n)$.

Assim, a função $\proc{Intercalacao}$, que inicializa a tabela, tem tempo:
\begin{align*}
    T(m, n) &= \sum_{i = 0}^m \sum_{j = 0}^n \Theta(1) + O(m n) \\
        &= \Theta((m + 1) (n + 1)) + O(m n) \\
        &= \Theta(m n)
\end{align*}

Para a complexidade de espaço, podemos ver que a função recursiva usa espaço adicional. Logo, o único armazenamento usado que depende da entrada é a criação da tabela $I$, com dimensões $(m + 1) \times (n + 1)$. Portanto, $E(m, n) = \Theta(m n)$.

    \endgroup

    \docline[]

    \section{2}
    \begingroup
        \subsection{a} \label{sec:2:a}

Suponha um percurso $P$ com custo $c(P)$ mínimo de $A_{i < n}$ até $A_n$ e um $A_{i < k < n}$ como parte do percurso. Logo, temos o subpercurso $R \subset P$ de $A_k$ até $A_n$. Considere também o percurso $R' \ne R$ de $A_k$ a $A_n$ com custo $c\left(R'\right)$ mínimo.

Assim, se $c\left(R'\right) < c(R)$ , então podemos formar o percurso total $P' = (P \setminus R) \cup R'$, tal que $c\left(P'\right) = c(P) - c(R) + c\left(R'\right) < c(P)$. Ou seja, $P$ não poderia ser considerado mínimo.

Logo, $c(R) \leq c\left(R'\right)$, ou seja, $R$ deve ser mínimo. Além disso, como $A_k$ era arbitrário, todo subpercurso de $P$ também deve ser mínimo.

~

Considere que $C_i$ é o menor custo de um percurso da aldeia $A_i$ para a $A_n$, sendo $1 \leq i \leq n$. Como os subpercursos são ótimos, podemos considerar todas as possíveis paradas $A_{i < j \leq n}$ com seus percursos ótimos, tomando o menor deles como parte do percurso ótimo de $A_i$. Note se $i = n$, então não resta nenhuma aldeia no percurso, ou seja, $C_i = 0$. Assim,
\begin{align*}
    C_i &= \min_{i < j \leq n}\left\{t_{i, j} + C_j\right\} \\
    C_n &= 0
\end{align*}

Nessa relação, o custo ótimo $C_i$ depende apenas dos custos $C_j$ das aldeias seguintes, com $i < j$, já que $t_{i, j} > 0$. Assim, podemos calcular os custo a partir da última aldeia $A_n$, sem necessidade de recursão ou memprização.

\itemdsep
\subsection{b}

\begin{codebox}
    \Procname{$\proc{Custo-Mínimo}(t, n)$}
    \li Seja $C[1 \twodots n]$ um novo vetor. \label{linha:a1:1}
    \li
    \li $C[n] \gets 0$ \label{linha:a1:2}
    \li \kw{para} $i = n - 1$ descendo \kw{até} $1$ \label{linha:for:1}
        \Do
    \li     $\id{cmin} \gets \infty$ \label{linha:a2:1}
    \li     \kw{para} $j = i + 1$ \kw{até} $n$ \label{linha:for:2}
            \Do
    \li         $\id{custo} \gets t[i][j] + C[j]$ \label{linha:a3:1}
    \li         $\id{cmin} \gets \min\left(\id{cmin},\, \id{custo}\right)$ \label{linha:a3:2}
            \End
    \li     $C[i] \gets \id{cmin}$ \label{linha:a2:2}
        \End
    \li
    \li \kw{retorna} $C[1]$ \label{linha:a1:3}
\end{codebox}

\itemdsep
\subsection{c}

Vamos considerar que as linhas \ref{linha:a1:1}, \ref{linha:a1:2} e \ref{linha:a1:3} executam em tempo constante $a_1$, que a \ref{linha:a2:1} e a \ref{linha:a2:2} executam em $a_2$ e que as \ref{linha:a3:1} e \ref{linha:a3:2} é em tempo $a_3$. Assim, podemos descrever o tempo de execução do algoritmo por:

\begin{align*}
    T(n) &= a_1 + \sum_{i = 1}^{n - 1}\left(a_2 + \sum_{j = i + 1}^n a_3\right) \\
    &= a_1 + a_2 (n - 1) + a_3 \sum_{i = 1}^{n - 1} n - a_3 \sum_{i = 1}^{n - 1} i \\
    &= a_1 + a_2 n - a_2 + a_3 n (n - 1) - a_3 \frac{n (n - 1)}{2} \\
    &= \frac{a_3}{2} n^2 + \frac{2 a_2 - a_3}{2} n + a_1 - a_2
\end{align*}

Ou seja, $T(n) \in \Theta\left(n^2\right)$, como requerido. Além disso, o único espaço adicional é do vetor $C$ de tamanho $n$. Então, a complexidade de espaço pode ser vista como:
\begin{align*}
    E(n) &= n + \Theta(1) = \Theta(n)
\end{align*}

    \endgroup

    \docline[]

    \section{3}
    \begingroup
        Para conseguir a sequência de aldeias, podemos guardar os índices que minimizam o cálculo do custo, dado pela recorrência do item \hyperref[sec:2:a]{2a)}. Assim, teremos um vetor $P$ de tamanho $n$ onde $C_i = t_{i, P_i} + C_{P_i}$, ou seja, $P_i$ é o índice da próxima aldeia no caminho ótimo partindo de $A_i$.

Na prática, esse índice poderia ser acumulado em uma variável $pmin$ assim como $cmin$ (linha \ref{linha:a2:1}), sendo trocado sempre que $cmin$ fosse atualizado (linha \ref{linha:a3:2}). Ou seja, teríamos uma nova linha com $\mlq\mlq pmin \gets j \;\mrq\mrq$ se $custo < cmin$, mantendo a invariante de que
\[
    cmin = \min_{i < k \leq j} \left\{t_{i, k} + C_k\right\} = t_{i, pmin} + C_{pmin}
\]
Ao final do laço da linha \ref{linha:for:2}, teremos então que $P_i = pmin$, além de $C_i = cmin$, como já acontecia na linha \ref{linha:a2:2}.

Com esse vetor $P$, podemos montar o caminho ótimo $M$ partindo de $A_1$. Assim, teremos $M_1 = P_1$ e $M_{i + 1} = P_{M_i}$, até encontrar um valor final $1 \leq f \leq n$ em que $M_f = n$, ou seja, a última aldeia. Esse caminho $M$ de tamanho $f$ seria o resultado do algoritmo.

% \begin{codebox}
%     \Procname{$\proc{Custo-Mínimo}(t, n)$}
%     \li Seja $C[1 \twodots n]$ um novo vetor.
%     \li Seja $P[1 \twodots n - 1]$ um novo vetor.
%     \li
%     \li $C[n] \gets 0$
%     \li \kw{para} $i = n - 1$ descendo \kw{até} $1$
%         \Do
%     \li     $\id{cmin} \gets +\infty$
%     \li     $\id{idmin} \gets -1$
%     \li     \kw{para} $j = i + 1$ \kw{até} $n$
%             \Do
%     \li         $\id{custo} \gets t[i][j] + C[j]$
%     \li         \kw{se} $\id{custo} < \id{cmin}$
%                 \Then
%     \li             $\id{cmin} \gets \id{custo}$
%     \li             $\id{idmin} \gets \id{j}$
%                 \End
%             \End
%     \li     $C[i] \gets \id{cmin}$
%     \li     $P[i] \gets \id{idmin}$
%         \End
%     \li
%     \li $\id{fim} = 0$
%     \li $i = 1$
%     \li \kw{enquanto} $i < n$
%         \Do
%     \li     $i \gets P[i]$
%     \li     $\id{fim} \gets \id{fim} + 1$
%     \li     $P[\id{fim}] \gets i$
%     \li     $i \gets i + 1$
%         \End
%     \li
%     \li \kw{retorna} $P[1 \twodots \id{fim}]$
% \end{codebox}

    \endgroup

    \docline[]

\end{document}
