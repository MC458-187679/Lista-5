\subsection{a} \label{sec:2:a}

Suponha um percurso $P$ com custo $c(P)$ mínimo de $A_{i < n}$ até $A_n$ e um $A_{i < k < n}$ como parte do percurso. Logo, temos o subpercurso $R \subset P$ de $A_k$ até $A_n$. Considere também o percurso $R' \ne R$ de $A_k$ a $A_n$ com custo $c\left(R'\right)$ mínimo.

Assim, se $c\left(R'\right) < c(R)$ , então podemos formar o percurso total $P' = (P \setminus R) \cup R'$, tal que $c\left(P'\right) = c(P) - c(R) + c\left(R'\right) < c(P)$. Ou seja, $P$ não poderia ser considerado mínimo.

Logo, $c(R) \leq c\left(R'\right)$, ou seja, $R$ deve ser mínimo. Além disso, como $A_k$ era arbitrário, todo subpercurso de $P$ também deve ser mínimo.

~

Considere que $C_i$ é o menor custo de um percurso da aldeia $A_i$ para a $A_n$, sendo $1 \leq i \leq n$. Como os subpercursos são ótimos, podemos considerar todas as possíveis paradas $A_{i < j \leq n}$ com seus percursos ótimos, tomando o menor deles como parte do percurso ótimo de $A_i$. Note se $i = n$, então não resta nenhuma aldeia no percurso, ou seja, $C_i = 0$. Assim,
\begin{align*}
    C_i &= \min_{i < j \leq n}\left\{t_{i, j} + C_j\right\} \\
    C_n &= 0
\end{align*}

Nessa relação, o custo ótimo $C_i$ depende apenas dos custos $C_j$ das aldeias seguintes, com $i < j$, já que $t_{i, j} > 0$. Assim, podemos calcular os custo a partir da última aldeia $A_n$, sem necessidade de recursão ou memprização.

\itemdsep
\subsection{b}

\begin{codebox}
    \Procname{$\proc{Custo-Mínimo}(t, n)$}
    \li Seja $C[1 \twodots n]$ um novo vetor. \label{linha:a1:1}
    \li
    \li $C[n] \gets 0$ \label{linha:a1:2}
    \li \kw{para} $i = n - 1$ descendo \kw{até} $1$ \label{linha:for:1}
        \Do
    \li     $\id{cmin} \gets \infty$ \label{linha:a2:1}
    \li     \kw{para} $j = i + 1$ \kw{até} $n$ \label{linha:for:2}
            \Do
    \li         $\id{custo} \gets t[i][j] + C[j]$ \label{linha:a3:1}
    \li         $\id{cmin} \gets \min\left(\id{cmin},\, \id{custo}\right)$ \label{linha:a3:2}
            \End
    \li     $C[i] \gets \id{cmin}$ \label{linha:a2:2}
        \End
    \li
    \li \kw{retorna} $C[1]$ \label{linha:a1:3}
\end{codebox}

\itemdsep
\subsection{c}

Vamos considerar que as linhas \ref{linha:a1:1}, \ref{linha:a1:2} e \ref{linha:a1:3} executam em tempo constante $a_1$, que a \ref{linha:a2:1} e a \ref{linha:a2:2} executam em $a_2$ e que as \ref{linha:a3:1} e \ref{linha:a3:2} é em tempo $a_3$. Assim, podemos descrever o tempo de execução do algoritmo por:

\begin{align*}
    T(n) &= a_1 + \sum_{i = 1}^{n - 1}\left(a_2 + \sum_{j = i + 1}^n a_3\right) \\
    &= a_1 + a_2 (n - 1) + a_3 \sum_{i = 1}^{n - 1} n - a_3 \sum_{i = 1}^{n - 1} i \\
    &= a_1 + a_2 n - a_2 + a_3 n (n - 1) - a_3 \frac{n (n - 1)}{2} \\
    &= \frac{a_3}{2} n^2 + \frac{2 a_2 - a_3}{2} n + a_1 - a_2
\end{align*}

Ou seja, $T(n) \in \Theta\left(n^2\right)$, como requerido. Além disso, o único espaço adicional é do vetor $C$ de tamanho $n$. Então, a complexidade de espaço pode ser vista como:
\begin{align*}
    E(n) &= n + \Theta(1) = \Theta(n)
\end{align*}
