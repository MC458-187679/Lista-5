\subsection{a} Considere a operação ``$\oplus$'' como a concatenação de cadeias e a notação de prefixo adotada em aula.

\begin{theorem}[subestrutura ótima]
    Sejam $n$ e $m$ inteiros positivos e considere as cadeias de caracteres de $\Sigma$ sendo $X = \langle x_1, \ldots, x_m \rangle$, $Y = \langle y_1, \ldots, y_n \rangle$ e $Z = \langle z_1, \ldots, z_{m + n} \rangle$. Então, $Z$ é uma intercalação de $X$ e $Y$ se e somente se uma das condições a seguir for atendida:

    1. $x_m = z_{n + m}$ ~e~ $Z_{m + n - 1}$ é intercalação $X_{m - 1}$ e $Y$.

    2. $y_n = z_{n + m}$ ~e~ $Z_{m + n - 1}$ é intercalação $X$ e $Y_{n - 1}$.
\end{theorem}

\begin{proof}[\textbf{Demonstração}]
    Considere $Z' = Z_{m + n - 1}$, $X' = X_{m - 1}$ e $Y' = Y_{n - 1}$.

    ~

    ($\rightarrow$) Considere que $Z$ é uma intercalação de $X$ e $Y$ e suponha ainda que a condição 1 seja falsa. Então, temos os seguintes casos

    \begin{casos}
        \item $x_m \ne z_{m + n}$. Assim, se $y_n \ne z_{m + n}$, então $Z$ não poderia ser uma intercalação de $X$ e $Y$, logo, $y_n = z_{m + n}$. Suponha agora que $Z'$ não é uma intercalação de $X$ e $Y'$, ou seja, não é possível decompor $Z'$ em subcadeias disjuntas iguais a $X$ e $Y'$.

        Para que exista uma subcadeia $S$ de $Z = Z' \oplus \langle z_{m + n} \rangle$ tal que $S = Y' \oplus \langle y_n \rangle = Y$, temos que $z_{m + n}$ deverá ser parte de $S$. Se $X$ não é subcadeia de $Z'$, como $z_{m + n}$ seria parte da subcadeia igual $Y$, então $X$ também não seria subcadeia de $Z$. Por outro lado, se $Y'$ não é subcadeia de $Z'$, então $Y = Y' \oplus \langle y_n \rangle$ também não poderá ser subcadeia de $Z$.

        Portanto, temos que $Z'$ deve ser uma intercalação de $X$ e $Y'$, ou seja, a condição 2 deve ser atendida.

        \item $x_m = z_{m + n}$ e $Z'$ não é intercalação de $X'$ e $Y$. Suponha que $z_{m + n}$ é parte da subcadeia $S$ de $Z$ que é igual a $X$. No entanto, como $Z'$ não é intercalação de $X'$ e $Y$, temos que $Z = Z' \oplus \langle z_{m + n} \rangle$ não pode ser intercalação de $X = X' \oplus \langle x_m \rangle = S$ e $Y$. Portanto, $z_{m + n}$ deve ser parte de $Y$.

        Como $z_{m + n}$ é o último elemento de $Z$, então $y_n = z_{m + n}$. Além disso, assim como mostrado no caso anterior, como $z_{m + n}$ é parte da subcadeia igual a $Y$, então temos que $Z'$ deve ser uma intercalação de $X$ e $Y'$.
    \end{casos}

    ~

    Por fim, como os casos são exaustivos, temos que se a condição 1 é falsa, a condição 2 deve ser verdade. Logo, para que $Z$ seja a intercalação proposta, uma das condições deve ser atendida.

    ~

    ($\leftarrow$) Considere agora que a condição 1 é verdade. Como $Z = Z' \oplus \langle z_{m + n} \rangle$, $X' = X \oplus \langle x_m \rangle$ e $z_{m + n} = x_m$, então as operações de concatenação mantêm a propriedade da intercalação. Portanto, $Z$ é uma intercalação de $X$ e $Y$.

    A condição 2 é similar.
\end{proof}

A partir da demonstração acima e considerando os casos onde $X$ ou $Y$ é vazio, podemos ver que a recorrência para verificação de intercalação será dada por:
\begin{align*}
    \varnothing &\text{ é intercalação de } \varnothing \text{ e } \varnothing \\
    Z \text{ é intercalação de } \varnothing \text{ e } Y &~\bm{\leftrightarrow}~ z_n = y_n ~\text{ \textbf{e} }~ Z_{n - 1} \text{ é intercalação de } \varnothing \text{ e } Y_{n - 1} \\
    Z \text{ é intercalação de } X \text{ e } \varnothing &~\bm{\leftrightarrow}~ z_m = x_m ~\text{ \textbf{e} }~ Z_{m - 1} \text{ é intercalação de } X_{m - 1} \text{ e } \varnothing \\
    Z \text{ é intercalação de } X \text{ e } Y &~\bm{\leftrightarrow} \begin{cases}
        z_{m + n} = x_m \text{ \textbf{e} } Z_{m + n - 1} \text{ é intercalação de } X_{m - 1} \text{ e } Y &\text{, ou} \\
        z_{m + n} = y_n \text{ \textbf{e} } Z_{m + n - 1} \text{ é intercalação de } X \text{ e } Y_{n - 1}
    \end{cases}
\end{align*}

\itemdsep

\subsection{b} Considere $\const{VERDADEIRO}$ como 1, $\const{FALSO}$ como 0 e $\const{NULO}$ como uma constante diferente de 1 e de 0.

\def\And{\mbox{ \kw{e} }}
\def\Or{\mbox{ \kw{ou} }}

\begin{codebox}
    \Procname{$\proc{Intercalacao}(X, m, Y, n, Z)$}
    \li Seja $I[0 \twodots m][0 \twodots n]$ uma matriz
    \li
    \li \kw{para} $i = 0$ \kw{até} $m$
        \Do
    \li     \kw{para} $j = 0$ \kw{até} $n$
            \Do
    \li         $I[i][j] \gets \const{NULO}$
            \End
        \End
    \li
    \li \kw{retorna} $\proc{Intercalacao-Rec}(X, m, Y, n, Z, I)$
\end{codebox}

\begin{codebox}
    \Procname{$\proc{Intercalacao-Rec}(X, m, Y, n, Z, I)$}
    \li \kw{se} $I[m][n] \ne \const{NULO}$
        \Then
    \li     \kw{retorna} $I[m][n]$
        \End
    \li
    \li \kw{se} $m \isequal 0 \And n \isequal 0$
        \Then
    \li     $I[0][0] \gets \const{VERDADEIRO}$
        \End
    \li \kw{senão}
        \Then
    \li     $I[m][n] \gets \const{FALSO}$
    \li     \kw{se} $m \ne 0 \And X[m] \isequal Z[m + n]$
            \Then
    \li         $I[m][n] \gets I[m][n] \Or \proc{Intercalacao-Rec}(X, m - 1, Y, n, Z, I)$
            \End
    \li     \kw{se} $n \ne 0 \And Y[n] \isequal Z[m + n]$
            \Then
    \li         $I[m][n] \gets I[m][n] \Or \proc{Intercalacao-Rec}(X, m, Y, n-1, Z, I)$
            \End
        \End
    \li
    \li \kw{retorna} $I[m][n]$
\end{codebox}

% \itemdsep
\newpage

\subsection{c}

Considerando $m'$ e $n'$ as entradas das chamadas recursivas de $\proc{Intercalao-Rec}$, é fácil notar que $0 \leq m' \leq m$ e $0 \leq n' \leq n$. Além disso, cada nível da recursão faz no máximo duas novas chamadas, ou seja, não cresce com o valor de $m$ ou $n$.

Logo, toda chamada leva tempo constante a partir do momento que a matriz $I$ de memorização está preenchida. Portanto, a complexidade da função recursiva é limitada por $O\left((m + 1) (n + 1)\right) = O(m n)$.

Assim, a função $\proc{Intercalacao}$, que inicializa a tabela e a recursão, tem tempo:
\begin{align*}
    T(m, n) &= \sum_{i = 0}^m \sum_{j = 0}^n \Theta(1) + O(m n) \\
        &= \Theta((m + 1) (n + 1)) + O(m n) \\
        &= \Theta(m n)
\end{align*}

Para a complexidade de espaço, podemos ver que a função recursiva usa espaço constante em cada chamada. Como as chamadas são limitadas pela tabela de memorização, então o espaço utilizado é $O(m n)$. Para a função final, a criação da tabela $I$, com dimensões $(m + 1) \times (n + 1)$, também determina a complexidade. Portanto, $E(m, n) = \Theta(m n)$.
